% uWaterloo Thesis Template for LaTeX 
% Last Updated May 24, 2011 by Stephen Carr, IST Client Services
% FOR ASSISTANCE, please send mail to rt-IST-CSmathsci@ist.uwaterloo.ca

% Effective October 2006, the University of Waterloo 
% requires electronic thesis submission. See the uWaterloo thesis regulations at
% http://www.grad.uwaterloo.ca/Thesis_Regs/thesistofc.asp.

% DON'T FORGET TO ADD YOUR OWN NAME AND TITLE in the "hyperref" package
% configuration below. THIS INFORMATION GETS EMBEDDED IN THE PDF FINAL PDF DOCUMENT.
% You can view the information if you view Properties of the PDF document.

% Many faculties/departments also require one or more printed
% copies. This template attempts to satisfy both types of output. 
% It is based on the standard "book" document class which provides all necessary 
% sectioning structures and allows multi-part theses.

% DISCLAIMER
% To the best of our knowledge, this template satisfies the current uWaterloo requirements.
% However, it is your responsibility to assure that you have met all 
% requirements of the University and your particular department.
% Many thanks to the feedback from many graduates that assisted the development of this template.

% -----------------------------------------------------------------------

% By default, output is produced that is geared toward generating a PDF 
% version optimized for viewing on an electronic display, including 
% hyperlinks within the PDF.
 
% E.g. to process a thesis called "mythesis.tex" based on this template, run:

% pdflatex mythesis	-- first pass of the pdflatex processor
% bibtex mythesis	-- generates bibliography from .bib data file(s) 
% pdflatex mythesis	-- fixes cross-references, bibliographic references, etc
% pdflatex mythesis	-- fixes cross-references, bibliographic references, etc

% If you use the recommended LaTeX editor, Texmaker, you would open the mythesis.tex
% file, then click the pdflatex button. Then run BibTeX (under the Tools menu).
% Then click the pdflatex button two more times. If you have an index as well,
% you'll need to run MakeIndex from the Tools menu as well, before running pdflatex
% the last two times.

% N.B. The "pdftex" program allows graphics in the following formats to be
% included with the "\includegraphics" command: PNG, PDF, JPEG, TIFF
% Tip 1: Generate your figures and photos in the size you want them to appear
% in your thesis, rather than scaling them with \includegraphics options.
% Tip 2: Any drawings you do should be in scalable vector graphic formats:
% SVG, PNG, WMF, EPS and then converted to PNG or PDF, so they are scalable in
% the final PDF as well.
% Tip 3: Photographs should be cropped and compressed so as not to be too large.

% To create a PDF output that is optimized for double-sided printing: 
%
% 1) comment-out the \documentclass statement in the preamble below, and
% un-comment the second \documentclass line.
%
% 2) change the value assigned below to the boolean variable
% "PrintVersion" from "false" to "true".

% --------------------- Start of Document Preamble -----------------------

% Specify the document class, default style attributes, and page dimensions
% For hyperlinked PDF, suitable for viewing on a computer, use this:
\documentclass[letterpaper,12pt,titlepage,oneside,final]{book}
 
% For PDF, suitable for double-sided printing, change the PrintVersion variable below
% to "true" and use this \documentclass line instead of the one above:
%\documentclass[letterpaper,12pt,titlepage,openright,twoside,final]{book}

% Some LaTeX commands I define for my own nomenclature.
% If you have to, it's better to change nomenclature once here than in a 
% million places throughout your thesis!
\newcommand{\package}[1]{\textbf{#1}} % package names in bold text
\newcommand{\cmmd}[1]{\textbackslash\texttt{#1}} % command name in tt font 
\newcommand{\href}[1]{#1} % does nothing, but defines the command so the
    % print-optimized version will ignore \href tags (redefined by hyperref pkg).
%\newcommand{\texorpdfstring}[2]{#1} % does nothing, but defines the command
% Anything defined here may be redefined by packages added below...

% This package allows if-then-else control structures.
\usepackage{ifthen}
\newboolean{PrintVersion}
\setboolean{PrintVersion}{false} 
% CHANGE THIS VALUE TO "true" as necessary, to improve printed results for hard copies
% by overriding some options of the hyperref package below.

%\usepackage{nomencl} % For a nomenclature (optional; available from ctan.org)
\usepackage{amsmath,amssymb,amstext} % Lots of math symbols and environments
\usepackage[pdftex]{graphicx} % For including graphics N.B. pdftex graphics driver 

% Hyperlinks make it very easy to navigate an electronic document.
% In addition, this is where you should specify the thesis title
% and author as they appear in the properties of the PDF document.
% Use the "hyperref" package 
% N.B. HYPERREF MUST BE THE LAST PACKAGE LOADED; ADD ADDITIONAL PKGS ABOVE
\usepackage[pdftex,letterpaper=true,pagebackref=false]{hyperref} % with basic options
		% N.B. pagebackref=true provides links back from the References to the body text. This can cause trouble for printing.
\hypersetup{
    plainpages=false,       % needed if Roman numbers in frontpages
    pdfpagelabels=true,     % adds page number as label in Acrobat's page count
    bookmarks=true,         % show bookmarks bar?
    unicode=false,          % non-Latin characters in Acrobat’s bookmarks
    pdftoolbar=true,        % show Acrobat’s toolbar?
    pdfmenubar=true,        % show Acrobat’s menu?
    pdffitwindow=false,     % window fit to page when opened
    pdfstartview={FitH},    % fits the width of the page to the window
    pdftitle={Biologically Inspired Adaptive Control of Quadcopter Flight},    % title
    pdfauthor={Brent Komer},    % author
%    pdfsubject={Subject},  % subject: CHANGE THIS TEXT! and uncomment this line
%    pdfkeywords={keyword1} {key2} {key3}, % list of keywords, and uncomment this line if desired
    pdfnewwindow=true,      % links in new window
    colorlinks=true,        % false: boxed links; true: colored links
    linkcolor=blue,         % color of internal links
    citecolor=green,        % color of links to bibliography
    filecolor=magenta,      % color of file links
    urlcolor=cyan           % color of external links
}
\ifthenelse{\boolean{PrintVersion}}{   % for improved print quality, change some hyperref options
\hypersetup{	% override some previously defined hyperref options
%    colorlinks,%
    citecolor=black,%
    filecolor=black,%
    linkcolor=black,%
    urlcolor=black}
}{} % end of ifthenelse (no else)

% Setting up the page margins...
% uWaterloo thesis requirements specify a minimum of 1 inch (72pt) margin at the
% top, bottom, and outside page edges and a 1.125 in. (81pt) gutter
% margin (on binding side). While this is not an issue for electronic
% viewing, a PDF may be printed, and so we have the same page layout for
% both printed and electronic versions, we leave the gutter margin in.
% Set margins to minimum permitted by uWaterloo thesis regulations:
\setlength{\marginparwidth}{0pt} % width of margin notes
% N.B. If margin notes are used, you must adjust \textwidth, \marginparwidth
% and \marginparsep so that the space left between the margin notes and page
% edge is less than 15 mm (0.6 in.)
\setlength{\marginparsep}{0pt} % width of space between body text and margin notes
\setlength{\evensidemargin}{0.125in} % Adds 1/8 in. to binding side of all 
% even-numbered pages when the "twoside" printing option is selected
\setlength{\oddsidemargin}{0.125in} % Adds 1/8 in. to the left of all pages
% when "oneside" printing is selected, and to the left of all odd-numbered
% pages when "twoside" printing is selected
\setlength{\textwidth}{6.375in} % assuming US letter paper (8.5 in. x 11 in.) and 
% side margins as above
\raggedbottom

% The following statement specifies the amount of space between
% paragraphs. Other reasonable specifications are \bigskipamount and \smallskipamount.
\setlength{\parskip}{\medskipamount}

% The following statement controls the line spacing.  The default
% spacing corresponds to good typographic conventions and only slight
% changes (e.g., perhaps "1.2"), if any, should be made.
\renewcommand{\baselinestretch}{1} % this is the default line space setting

% By default, each chapter will start on a recto (right-hand side)
% page.  We also force each section of the front pages to start on 
% a recto page by inserting \cleardoublepage commands.
% In many cases, this will require that the verso page be
% blank and, while it should be counted, a page number should not be
% printed.  The following statements ensure a page number is not
% printed on an otherwise blank verso page.
\let\origdoublepage\cleardoublepage
\newcommand{\clearemptydoublepage}{%
  \clearpage{\pagestyle{empty}\origdoublepage}}
\let\cleardoublepage\clearemptydoublepage

% Allow chapter references to be capitalized
\newcommand{\Chapref}[1]{Chapter~\ref{#1}}

\newcommand{\Tabref}[1]{Table~\ref{#1}}

% Allow subfigures
\usepackage{subcaption}

% Find the figures stored in this folder
\graphicspath{./figures/}

% Uncomment these to remove all figures and tables for a print version for editing
%\usepackage{comment}
%\excludecomment{figure}
%\excludecomment{table}
%\let\endfigure\relax
%\let\endtable\relax

%======================================================================
%   L O G I C A L    D O C U M E N T -- the content of your thesis
%======================================================================
\begin{document}

% For a large document, it is a good idea to divide your thesis
% into several files, each one containing one chapter.
% To illustrate this idea, the "front pages" (i.e., title page,
% declaration, borrowers' page, abstract, acknowledgements,
% dedication, table of contents, list of tables, list of figures,
% nomenclature) are contained within the file "uw-ethesis-frontpgs.tex" which is
% included into the document by the following statement.
%----------------------------------------------------------------------
% FRONT MATERIAL
%----------------------------------------------------------------------
% T I T L E   P A G E
% -------------------
% Last updated May 24, 2011, by Stephen Carr, IST-Client Services
% The title page is counted as page `i' but we need to suppress the
% page number.  We also don't want any headers or footers.
\pagestyle{empty}
\pagenumbering{roman}

% The contents of the title page are specified in the "titlepage"
% environment.
\begin{titlepage}
        \begin{center}
        \vspace*{1.0cm}

        \Huge
        {\bf University of Waterloo E-Thesis Template for \LaTeX }

        \vspace*{1.0cm}

        \normalsize
        by \\

        \vspace*{1.0cm}

        \Large
        Pat Neugraad \\

        \vspace*{3.0cm}

        \normalsize
        A thesis \\
        presented to the University of Waterloo \\ 
        in fulfillment of the \\
        thesis requirement for the degree of \\
        Master of Science \\
        in \\
        Zoology \\

        \vspace*{2.0cm}

        Waterloo, Ontario, Canada, 2007 \\

        \vspace*{1.0cm}

        \copyright\ Pat Neugraad 2007 \\
        \end{center}
\end{titlepage}

% The rest of the front pages should contain no headers and be numbered using Roman numerals starting with `ii'
\pagestyle{plain}
\setcounter{page}{2}

\cleardoublepage % Ends the current page and causes all figures and tables that have so far appeared in the input to be printed.
% In a two-sided printing style, it also makes the next page a right-hand (odd-numbered) page, producing a blank page if necessary.
 


% D E C L A R A T I O N   P A G E
% -------------------------------
  % The following is the sample Delaration Page as provided by the GSO
  % December 13th, 2006.  It is designed for an electronic thesis.
  \noindent
I hereby declare that I am the sole author of this thesis. This is a true copy of the thesis, including any required final revisions, as accepted by my examiners.

  \bigskip
  
  \noindent
I understand that my thesis may be made electronically available to the public.

\cleardoublepage
%\newpage

% A B S T R A C T
% ---------------

\begin{center}\textbf{Abstract}\end{center}

This is the abstract.

Vulputate minim vel consequat praesent at vel iusto et, ex delenit, esse euismod luptatum augue ut sit et eu vel augue autem feugiat, quis ad dolore. Nulla vel, laoreet lobortis te commodo elit qui aliquam enim ex iriure ea ullamcorper nostrud lorem, lorem laoreet eu ex ut vel in zzril wisi quis. Nisl in autem praesent dignissim, sit vel aliquam at te, vero dolor molestie consequat.

Tation iriure sed wisi feugait odio dolore illum duis in accumsan velit illum consequat consequat ipsum molestie duis duis ut ullamcorper. Duis exerci odio blandit vero dolore eros odio amet et nisl in nostrud consequat iusto eum suscipit autem vero. Iusto dolore exerci, ut erat ex, magna in facilisis duis amet feugait augue accumsan zzril delenit aliquip dignissim at. Nisl molestie nibh, vulputate feugait nibh luptatum ea delenit nostrud dolore minim veniam odio volutpat delenit nulla accumsan eum vero ullamcorper eum. Augue velit veniam, dolor, exerci ea feugiat nulla molestie, veniam nonummy nulla dolore tincidunt, consectetuer dolore nulla ipsum commodo.

At nostrud lorem, lorem laoreet eu ex ut vel in zzril wisi. Suscipit consequat in autem praesent dignissim, sit vel aliquam at te, vero dolor molestie consequat eros tation facilisi diam dolor. Odio luptatum dolor in facilisis et facilisi et adipiscing suscipit eu iusto praesent enim, euismod consectetuer feugait duis. Odio veniam et iriure ad qui nonummy aliquip at qui augue quis vel diam, nulla. Autem exerci tation iusto, hendrerit et, tation esse consequat ut velit te dignissim eu esse eros facilisis lobortis, lobortis hendrerit esse dignissim nisl. Nibh nulla minim vel consequat praesent at vel iusto et, ex delenit, esse euismod luptatum.

Ut eum vero ullamcorper eum ad velit veniam, dolor, exerci ea feugiat nulla molestie, veniam nonummy nulla. Elit tincidunt, consectetuer dolore nulla ipsum commodo, ut, at qui blandit suscipit accumsan feugiat vel praesent. In dolor, ea elit suscipit nisl blandit hendrerit zzril. Sit enim, et dolore blandit illum enim duis feugiat velit consequat iriure sed wisi feugait odio dolore illum duis. Et accumsan velit illum consequat consequat ipsum molestie duis duis ut ullamcorper nulla exerci odio blandit vero dolore eros odio amet et.

In augue quis vel diam, nulla dolore exerci tation iusto, hendrerit et, tation esse consequat ut velit. Duis dignissim eu esse eros facilisis lobortis, lobortis hendrerit esse dignissim nisl illum nulla minim vel consequat praesent at vel iusto et, ex delenit, esse euismod. Nulla augue ut sit et eu vel augue autem feugiat, quis ad dolore te vel, laoreet lobortis te commodo elit qui aliquam enim ex iriure. Ut ullamcorper nostrud lorem, lorem laoreet eu ex ut vel in zzril wisi quis consequat in autem praesent dignissim, sit vel. Dolore at te, vero dolor molestie consequat eros tation facilisi diam. Feugait augue luptatum dolor in facilisis et facilisi et adipiscing suscipit eu iusto praesent enim, euismod consectetuer feugait duis vulputate veniam et.

Ad eros odio amet et nisl in nostrud consequat iusto eum suscipit autem vero enim dolore exerci, ut. Esse ex, magna in facilisis duis amet feugait augue accumsan zzril. Lobortis aliquip dignissim at, in molestie nibh, vulputate feugait nibh luptatum ea delenit nostrud dolore minim veniam odio. Euismod delenit nulla accumsan eum vero ullamcorper eum ad velit veniam. Quis, exerci ea feugiat nulla molestie, veniam nonummy nulla. Elit tincidunt, consectetuer dolore nulla ipsum commodo, ut, at qui blandit suscipit accumsan feugiat vel praesent.

Dolor zzril wisi quis consequat in autem praesent dignissim, sit vel aliquam at te, vero. Duis molestie consequat eros tation facilisi diam dolor augue. Dolore dolor in facilisis et facilisi et adipiscing suscipit eu iusto praesent enim, euismod consectetuer feugait duis vulputate.

\cleardoublepage
%\newpage

% A C K N O W L E D G E M E N T S
% -------------------------------

\begin{center}\textbf{Acknowledgements}\end{center}

I would like to thank all the little people who made this possible.
\cleardoublepage
%\newpage

% D E D I C A T I O N
% -------------------

\begin{center}\textbf{Dedication}\end{center}

This is dedicated to the one I love.
\cleardoublepage
%\newpage

% T A B L E   O F   C O N T E N T S
% ---------------------------------
\renewcommand\contentsname{Table of Contents}
\tableofcontents
\cleardoublepage
\phantomsection
%\newpage

% L I S T   O F   T A B L E S
% ---------------------------
\addcontentsline{toc}{chapter}{List of Tables}
\listoftables
\cleardoublepage
\phantomsection		% allows hyperref to link to the correct page
%\newpage

% L I S T   O F   F I G U R E S
% -----------------------------
\addcontentsline{toc}{chapter}{List of Figures}
\listoffigures
\cleardoublepage
\phantomsection		% allows hyperref to link to the correct page
%\newpage

% L I S T   O F   S Y M B O L S
% -----------------------------
% To include a Nomenclature section
% \addcontentsline{toc}{chapter}{\textbf{Nomenclature}}
% \renewcommand{\nomname}{Nomenclature}
% \printglossary
% \cleardoublepage
% \phantomsection % allows hyperref to link to the correct page
% \newpage

% Change page numbering back to Arabic numerals
\pagenumbering{arabic}

 

%----------------------------------------------------------------------
% MAIN BODY
%----------------------------------------------------------------------
% Because this is a short document, and to reduce the number of files
% needed for this template, the chapters are not separate
% documents as suggested above, but you get the idea. If they were
% separate documents, they would each start with the \chapter command, i.e, 
% do not contain \documentclass or \begin{document} and \end{document} commands.
%======================================================================
\chapter{Introduction}
%======================================================================

\section{Motivation}

%[talk about humans being a very good control system, want to build a robotic system that leverages some of the abilities humans have, by building it with a biologically realistic neural simulator. Can also be run on low power neural hardware in the future]

Humans have an exceptional ability to be able to adapt to their surroundings.
In particular the human motor control system is able to compensate for changes in forces, torques, and inertial effects on the body.
For example, when picking up an object such as a hammer, the weight of the hammer will apply external forces to the hand.
This will change the dynamic properties of the hand and arm movements, yet the human motor control system is able to easily compensate for these changes and control their movement with the object.
%Skilfully manipulating an object requires
Even if an object has never been encountered before, the human brain is able to calculate the correct changes in timing and muscle tensions in order to skilfully manipulate the object. %For example, humans are able to use hammers with a variety of styles and weights
%When picking up a new object, there are a many unknown forces, torques, and inertial effects that this object will apply to the arm. 
%Despite all of these unknowns, humans are very good at manipulating objects and the necessary changes in timing and muscle tensions to do so are calculated with ease, even if the object has never been encountered before. 
This ability can be largely attributed to the plasticity of neural connections in the motor control area of the brain.

This ability for quick and easy adaptation to new dynamic properties of a system would be extremely useful in robotics. 
Applying similar methods of control that have been developed over millions of years of evolution in the brain to a robotic control system could result in major improvements. 
This is especially useful now that the demands of many robotic systems are now more general purpose than they were in the past. 
Robots started out mainly performing simple and repetitive tasks in stable environments, such as automation in manufacturing [[throw in some references]]. 
Now they are being used increasingly in more complex situations requiring a diverse amount of control, such as search and rescue missions, performing medical procedures, and assisting the elderly [[put references here]]. 
When the precise environment that the robot will operate in is not fully known, it is useful for any control system that the robot uses to be adaptable to those environments.

Another advantage the brain has when it comes to control, is that it uses very little power, about 20 Watts on average \cite{hart1975brain}.
% the motor is the majority of the power on a quadcopter, so neuromorphic hardware might not help much here
Hardware inspired by the brain is being designed to take advantage of this low power paradigm. 
This style of hardware, known as neuromorphic hardware, is typically massively parallel and runs in an analog rather than digital domain.

\section{Introduction}

%[basic characteristics, four rotors, rotating different ways]
%[something about vertical take off and landing]
%[light-weight]
%[very maneuverable]

Quadcopters are very versatile aerial vehicles, typically consisting of a central body with four upright rotors equally spaced around the body. 
This configuration allows them to be lightweight and simplifies construction. 
They also have the ability to take off and land vertically without external assistance, meaning they can be used in a lot more situations than aircraft that require a runway or other particular conditions to take off. 
They can also change directions fairly easily mid flight, and have the ability to hover at a particular location. Due to their low cost, light weight, and ease of use, they are an excellent aircraft to use for research purposes.

%[battery heaviest part, low power controller means it can last longer and/or be lighter, good candidate for neural control]
Despite all of these positive qualities, one of the main weaknesses of quadcopters is their short battery life. 
Small and simple quadcopters used by hobbyists typically last only 10 to 20 minutes before they run out of power [[put in source]]. 
The more expensive industrial grade quadcopters can last 1 to 2 hours before they need to be recharged [[check this]] [[put in source]], but this duration is still not long enough for some applications. 
For example, if a quadcopter is being used in a search and rescue mission it may need to remain flying for a long time without recharging, especially if it is operating in a remote area. 
Battery weight is one of the main issues hindering the maximum flight time of quadcopters. 
If a larger battery is used in attempts to increase the maximum flight time, the quadcopter will be heavier and therefore require more power to fly, which in turn will drain the battery faster. 
Two possible solutions to this problem are lighter batteries and more energy-efficient operation.
The latter can be achieved by the use of neuromorphic hardware to run the flight control system.

%[something about how adaptive control is awesome]
In addition to the low power consumption of neuromorphic hardware, they also have the advantage of using various types of neural algorithms that can be extremely useful for control.
In particular, the learning capabilities of brain-like algorithms can be used to develop controllers that are able to adapt to unknown environments with non-linear dynamics. 
This thesis explores an adaptive control algorithm for quadcopter flight constructed using simulated biologically plausible neurons.

%[talk about what each section of the thesis will talk about]
\Chapref{chap:background} gives an overview of how quadcopters are able to fly, followed by a section detailing the mathematics underlying the dynamics of a quadcopter system. 
Following the mathematical characterization is a description of how the quadcopter is modelled in a simulation environment.
\Chapref{chap:control} gives an overview of the control theory used in developing the control system, including standard PID control, adaptive control, and adaptive control in a simulated biological neural system. 
\Chapref{chap:implementation} covers the initial design of the controller model, as well as many of the iterations of improvements leading to the final design.
\Chapref{chap:analysis} describes the set of experiments performed and metrics used to quantify performance of the various controllers. 
The final section discusses the results and outlines areas of future work.

%======================================================================
\chapter{Background Information} \label{chap:background}
%======================================================================

\section{Flight Control}

%[mention the six states: x,y,z,roll,pitch,yaw -> have a diagram to explain them]
%[different modes of flight control (up/down, horizontal, rotate, hover), diagrams for each]
%[thrust is always perpendicular to the body]
%[two ways to do it, rotor aligned axis, and 45 degrees from that]

The system state of a quadcopter is six dimensional: three for position (x, y, and z) and three for orientation (roll, pitch, and yaw). 
There are four control inputs, which are typically taken to be the rotational velocity of each of the four rotors. 
For a physical quadcopter, these inputs are the voltages applied to each of these rotors.
A transformation in these voltages can be obtained using the physical rotor parameters to estimate the rotor velocity.
For simplicity of the simulation, the velocity is used directly in this thesis. [continue...]

%[talk about perpendicular thrust, as well as torque on the body, and that is how each of the movements is done]
The rotors will always generate a thrust that is perpendicular to the body of the aircraft. 
In addition to this thrust, torque is generated by each rotor based on their spin direction  as well as their distance from the center of the body.
For the quadcopter to be able to control its orientation, half of the rotors spin clockwise and the other half spin counter-clockwise. 
The pairs diagonally opposite each other across the body spin in the same direction, allowing the torques to balance one another out and the quadcopter to maintain a steady orientation. 
By varying the relative speeds of each rotor, the quadcopter is able to create a net torque in a specific direction, causing a rotation about any axis.

A quadcopter performs four common actions to move around in its environment, with a distinct pattern of rotor actuation for each one: 
tilt forward/backward, tilt left/right, rotate, and move up/down. 
A quadcopter can perform any combination of these actions, and with different magnitudes of each. The relative rotor speeds required for each are shown in \autoref{fig:actions}.

\begin{figure}
\centering
\includegraphics[height=0.3\textheight]{./figures/QuadcopterMovements.JPG}
\caption{Four Primary Quadcopter Movements. Taken from \cite{harsha}}
\label{fig:actions}
\end{figure}


\section{Dynamics} \label{sec:dynamics}

%[go through all of the equations for dynamics]

The structure of a quadcopter is shown in \autoref{fig:FBD}, along with the coordinate frame and relevant forces and torques created by the rotors.
%TODO [fix this phrase a ton] 
Two important frames of reference are involved when studying quadcopters: the inertial frame is used to represent the state of the quadcopter relative to the outside world, and the body frame is fixed around the center of mass of the quadcopter and used to represent local effects on the quadcopter.
%Typically you are interested in the state of the quadcopter in the inertial frame, but the control of the quadcopter is easier to represent in the body frame.
Typically, the inertial frame is the focus when researchers are interested in the state of the quadcopter, but the body frame is more useful when considering control of the quadcopter.

%[put diagram around here]
\begin{figure}
\centering
\includegraphics[width=0.8\textwidth]{./figures/QuadcopterFBD.png}
\caption{The Inertial and Body Frames of a Quadcopter. Taken from \cite{luukkonen} } %TODO get a better caption
\label{fig:FBD}
\end{figure}

A rotation matrix \eqref{eq:rotation_matrix} can be used to convert coordinates from the body frame to the inertial frame. 
To convert the other way, from inertial frame to body frame, the inverse of the matrix is used. 
Since this matrix is orthogonal, its inverse is equal to its transpose.

%[rotation matrix here]
\begin{equation} \label{eq:rotation_matrix}
R = 
\begin{bmatrix}
C_{\psi}C_{\theta} & C_{\psi}S_{\theta}S_{\phi} - S_{\psi}C_{\phi} & C_{\psi}S_{\theta}C_{\phi} + S_{\psi}S_{\phi} \\
S_{\psi}C_{\theta} & S_{\psi}S_{\theta}S_{\phi} + C_{\psi}C_{\phi} & S_{\psi}S_{\theta}C_{\phi} - C_{\psi}S_{\phi} \\
-S_{\theta} & C_{\theta}S_{\phi} & C_{\theta}C_{\phi}
\end{bmatrix}
\end{equation}

The transformation matrices for the angular velocities between the inertial and body frames are shown in \eqref{eq:global_angular_velocity} and \eqref{eq:local_angular_velocity}.

%[matrices here]
\begin{equation} \label{eq:global_angular_velocity}
\dot{\eta} = W_{\eta}^{-1}\nu, 
\begin{bmatrix}
\dot{\phi} \\
\dot{\theta} \\
\dot{\psi}
\end{bmatrix}
=
\begin{bmatrix}
1 & S_{\phi}T_{\theta} & C_{\phi}T_{\theta} \\
0 & C_{\phi} & -S_{\phi} \\
0 & S_{\phi}/C_{\theta} & C_{\phi}/C_{\theta}
\end{bmatrix}
\begin{bmatrix}
p \\
q \\
r
\end{bmatrix}
\end{equation}

\begin{equation} \label{eq:local_angular_velocity}
\nu = W_{\eta}\dot{\eta},
\begin{bmatrix}
p \\
q \\
r
\end{bmatrix}
=
\begin{bmatrix}
1 & 0 & -S_{\theta} \\
0 & C_{\phi} & C_{\theta}S_{\phi} \\
0 & -S_{\phi} & C_{\theta}C_{\phi}
\end{bmatrix}
\begin{bmatrix}
\dot{\phi} \\
\dot{\theta} \\
\dot{\psi}
\end{bmatrix}
\end{equation}

The quadcopter is assumed to be symmetrical with equal length arms for each rotor. 
Aligning the arms along the body's x and y axes gives the diagonal inertia matrix in \eqref{eq:inertia_matrix}. 
Due to symmetry, $I_{xx} = I_{yy}$.

%[put inertial matrix here]
\begin{equation} \label{eq:inertia_matrix}
\begin{bmatrix}
I_{xx} & 0 & 0 \\
0 & I_{yy} & 0 \\
0 & 0 & I_{zz}
\end{bmatrix}
\end{equation}

The angular velocity of each rotor generates a thrust force perpendicular to the body frame, as well as a torque about the rotor axis. 
This relationship is shown in \eqref{eq:rotor_force} and \eqref{eq:rotor_torque}, 
where $\omega$ is the angular velocity of the $i$th rotor, $k$ is the lift constant, $b$ is the drag constant, and $I_{M}$ is the rotor's moment of inertia.

%[equations here]
\begin{equation} \label{eq:rotor_force}
f_{i} = kw_{i}^{2}
\end{equation}
\begin{equation} \label{eq:rotor_torque}
\tau_{M_{i}} = b\omega_{i}^{2} + I_{M}\dot{\omega_{i}}
\end{equation}

Typically the effect of the angular acceleration is considered small and is omitted in most analyses, so it will not be present in the remainder of the dynamics derivations.

Note that the forces and torques generated are always proportional to the square of the rotor's angular velocity, thus working with this term instead of the angular velocity itself is simpler. %[put this part under simulation instead]?

Combining the forces of all four rotors leads to \eqref{eq:thrust}, where $T$ is the thrust in the direction of the z-axis of the body. 
Combining the torques of all four rotors leads to \eqref{eq:torque}, where the vector tau represents the torques across each of the principal body axes ($\tau_{\phi}$ for roll, $\tau_{\theta}$ for pitch, and $\tau_{\psi}$ for yaw). 
The distance between the rotor and the center of mass of the quadcopter is denoted by $l$. 

%[put the two equations here]
\begin{equation} \label{eq:thrust}
T = \sum_{i=1}^{4} f_{i} = k\sum_{i=1}^{4} w^{2}_{i} , T_{B} = 
\begin{bmatrix}
0 \\
0 \\
T
\end{bmatrix}
\end{equation}
\begin{equation} \label{eq:torque}
\tau_{B} = 
\begin{bmatrix}
\tau_{\phi} \\
\tau_{\theta} \\
\tau_{\psi}
\end{bmatrix}
=
\begin{bmatrix}
lk(-w_{2}^{2}+w_{4}^{2}) \\
lk(-w_{1}^{2}+w_{3}^{2}) \\
\sum_{i=1}^{4}\tau_{M_{i}}
\end{bmatrix}
\end{equation}

%[[put some description explaining these equations here]]

%TODO have some equations which have the drag force in them, as this will be referenced later

\section{Simulation}

%[maybe move this section later in the thesis, possibly put pure python sim here and v-rep sim elsewhere (with implementation?)]
%[talk about V-REP, throw in a reference for it]
%[possibly mention MORSE and choosing VREP over it]
%[talk about pure python simulator built from dynamics equations]
%[VREP is used because it allows complex environments and sensors]

%should probably include somewhere what values were used for each of the 'parameters' in the simulation, i.e. mass, inertia, lengths of rotor arms, etc

The quadcopter model was validated and tested using a computer simulation environment. 
The advantage of starting with a simulation rather than going straight to physical hardware is that it is quicker and easier to prototype new control algorithms, and it is much less costly to do so. 
For this project I decided to use the Virtual Robotics Experimentation Platform (V-REP)\cite{vrep} as the physical simulation environment.
This open-source software package is free to use for educational purposes; 
contains models of various robotic platforms, including a model of a quadcopter; 
and allows the construction of intricate 3D virtual environments for the quadcopter to interact with, as well as many virtual sensors with which to equip the quadcopter, such as cameras.

%[possible bit about choosing the simulator]
Two other strong candidates for the physics simulator where MORSE [reference] and Gazebo [reference]. 
%[continue with a little blurb about them and why they were not chosen, and reference the comparison table]

%[possible table comparing V-REP/MORSE/Gazebo with pros and cons of each]

The simulation environment in V-REP was set up to be a large open space with small obstacles and walls along the ground. 
The quadcopter model was initialized to start in the center of the space, 0.5 meters above the ground. 
This model is the one provided with the V-REP software courtesy of Eric Rohmer and Lyall Randell. 
The quadcopter's target is a small semi-transparent green sphere. The goal of the control system is to make the quadcopter move to the target's location with the target's orientation. 
Throughout the environment there are various wind tunnels, which are zones of space which will exert an external force on the quadcopter in a particular direction if the quadcopter is within the zone. 
There are also zones that can exert various nonlinear forces on the quadcopter. 
For example, one zone could push the quadcopter in the z direction with a force proportional to the square of its horizontal velocity. 
In addition to these zones, the quadcopter can also pick up and drop off boxes of various mass. 
These wind zones and boxes are used to test how well the controller can adapt to changes in its dynamics.

%TODO put in a screenshot here?

\chapter{Control} \label{chap:control}

\section{PID Control}

%[go over standard control, talk about PD and PID systems]
%[put in a block diagram of a simple PID system -> maybe show spots where disturbances can enter]
%[talk about applying PD and PID control to a quadcopter]
%[mention limits of PID control that adaptive control is trying to solve]

The canonical method for building a control system is to use a PID controller. 
This is a closed loop controller that works by applying gains to an error signal and feeding the result as input to the plant (system being controlled). 
PID stands for Proportional, Integral, and Derivative, referring to the three types of gains used. 
One gain is applied directly to the error (proportional); a second gain is applied to the derivative of the error, and the third gain is applied to the integral of the error. 
Each of these gains serves a unique purpose, and the careful tuning of all three is required to get the desired performance of the controller. 
The proportional gain is the main driver of the system and responds directly to the error to bring the system towards its target point. 
The larger the gain, the faster this response will be.
However, using only this gain will not account for any inertia in the system and will often cause the entity being controlled to overshoot its target, especially when the gain is large. 
In some cases, the overshoot can be so large that the system becomes unstable and oscillates out of control.
Thus, the derivative gain is usually used in addition, because it is sensitive to the rate of change of the error, and attempts to bring this rate towards zero. 
The faster the system is moving towards its target state, the more this gain acts to slow it down, by effectively providing a form of damping and so reducing the amount of overshoot.
Depending on how this parameter is tuned, and the properties of the system at hand, it can remove overshooting entirely. 
Due to unmodelled disturbances or changing external inputs, there may be a steady state error in the system, wherein the controller has reached a stable state, yet there is still an error being measured. 
The integral term is designed to account for this steady state error by keeping a running sum of the error over time. 
Eventually, this integral error should be large enough that it will drive the system to close the steady state error gap.

\section{Adaptive Control} \label{sec:adaptive_control}

%[name a few other adaptive control methods and give references]
%[go over the slotine adaptive control stuff, throw in a bunch of references]
%[maybe throw in a bunch of the equations and derivations I did on paper]
%[mention getting the slotine stuff working on the python simulator]
%[talk about how Nengo can be used to do adaptive control with learning]

Sometimes the kinematics and dynamics of the system being controlled are unknown, or change over time in an unknown fashion. 
A controller that works well for the system initially may not be well suited when the system undergoes changes. 
In this situation, it is useful to have a controller that can adapt to these changes.

The foundation of adaptive control is based on parameter estimation. 
First, a mathematical model of the system to be controlled is generated based on physical laws. 
This model is typically of the form shown in \eqref{eq:physical_equation}, where $q$ is the vector of state variables, $M(q)$ is a mass/inertia matrix, $C(q,\dot{q})$ is the coriolis term [???], $g(q)$ is the gravitational force, and $\tau$ is a vector representing the input force/torque to the system.

\begin{equation} \label{eq:physical_equation}
M(q)\ddot{q} + C(q,\dot{q})\dot{q} + g(q) = \tau
\end{equation}

%TODO
%[[[[[a lot more stuff to go here, need to understand it more first]]]]]
The goal of the controller is to bring the system to a particular target state.
Typically the state as well as its derivative is desired to be controlled.
Thus, the second derivative of the state will be zero when the system has arrived at the target state.
Setting $\ddot{q}$ to be zero gives the relationship of the inputs to the rest of the state shown in \eqref{eq:input_equation}.

\begin{equation} \label{eq:input_equation}
\tau = C(q,\dot{q})\dot{q} + g(q)
\end{equation}

An estimate of $q$ and $\dot{q}$ can typically be measured, leaving the only potentially unknown quantities in the right-hand side of the equation to be the physical parameters of the system.
If these physical parameters are constant and linear with respect to the system state, the equation can be reorganized as in \eqref{eq:Y_equation}.
Here $\hat{\theta}$ is a an estimate of the vector of unknown constant parameters of the system and $Y(q,\dot{q})$ is a known matrix dependent on the system state.

\begin{equation} \label{eq:input_equation}
\tau = C(q,\dot{q})\dot{q} + g(q) = Y(q,\dot{q})\hat{\theta}
\end{equation}

Creating a mathematical model of a system with enough detail to account for everything is difficult. 
Assumptions and approximations must be made if the model is to be manageable. 
Moreover, external forces from the environment may influence the model, and their form may be unknown as the environment can be largely unknown. 
One way to overcome this problem is to use a set of basis functions as the model, and the weights applied to each element of the basis as the constant parameters. 
If the basis is designed such that it can represent any computable function to a reasonable degree of accuracy, it will be effective in the adaptive control problem. 
Gaussian basis functions are commonly used in adaptive control \cite{sanner1992gaussian}, but neural networks may be used as well \cite{barto1983neuronlike}.

\section{Neural Simulation}

%[brief description of the NEF and Nengo and how it can be used, mostly just put references here]
%put a reference to Travis' thesis as an example of how Nengo's learning can be used for motor control

% Maybe this section shouldn't be here at all, and just have a brief description at the start of the next section?

Simulation of biological neurons is carried out by the software package Nengo \cite{bekolay2013nengo}.
This software implements the algorithms in the Neural Engineering Framework (NEF) [[reference]]

\section{Adaptive Control in Nengo}

%[Brief overview of Nengo (can be made longer with subsections if need be)]
%[Talk about the PES rule, how it relates to the slotine stuff]
%[talk about how there is no need to tune the KI term with the PES rule, this adaptive I term is easier to use]

The adaptive control methods described above can be applied to a network built using Nengo. 
An ensemble of simulated neurons is used as the set of basis functions for the physical model, and the decoders of these neurons are used as the vector of unknown constant parameters. 
A learning rule known as the Prescribed Error Sensitivity (PES) rule is used to update the decoder values \cite{bekolay2013simultaneous}. 

This learning method works by first creating a connection from an ensemble of spiking neurons representing the state of the system to a node representing the output of the controller. 
This is known as the "learned connection" and can be initially set to perform any transformation, but is typically initialized for the output to be random or zero. 
If the designer has an approximation of what the final learned transformation should look like, they can set this as the initial transformation. 
Doing so will allow the system to converge to the final transformation more quickly.

The learned connection will be modulated by an error signal, which can come from anywhere in the network. 
The PES learning rule will attempt to reduce the error signal by changing the value of the decoders on the learned connection. 
The direction in which the decoder values change is dependent on the sign of the error. 
The magnitude of the change in decoder values at each time step is dependant on both the magnitude of the error and a learning-rate parameter. 
The learning rate is a dimensionless parameter that needs to be tuned for the specific application. 
It is dependant on the simulation time step, the number of neurons in the state population, as well as how responsive the model needs to be to changes. [[check to make sure this is true!!]] 
A larger learning rate will cause larger reactions to error, effectively making the system trust its current measurements more than historical ones. 
A smaller timestep means that these changes will occur more frequently, so the net change over time will be greater. 
A larger number of neurons means that the changes will be greater, as there will be more decoders changing. The overall transformation is a sum of these decoders. [[check to make sure this is true!!]].
%[[talk about differing dimensions in these populations?]]

A simple example of using the PES learning rule to control an inverted pendulum is shown in Figure XXX for clarification before employing the rule the more high dimensional case of a quadcopter controller. [[describe how the model works in the figure description, talk about the different dimensionalities of each population]]

%[put in a simple network diagram showing the PES rule, and possibly an equation with its math]



% Should a chapter on simulation go here??
%======================================================================
%\chapter{Simulation} \label{chap:simulation}
%======================================================================

%======================================================================
\chapter{Implementation} \label{chap:implementation}
%======================================================================

%TODO should the pure math adaptive controller be talked about here? Maybe put this as a subsection in adaptive control? This might make more sense to be after the basic controller model is introduced, but before all of the iterations
\section{Derivation of Non-Neural Adaptive Controller} %TODO pick a better title
%this section will be heavy on math, and go through how the adaptive python model was formed (from the equations on all of those sheets of paper)

Based on the dynamics equations in \autoref{sec:dynamics} and the adaptive control theory presented in \autoref{sec:adaptive_control}, an adaptive controller for a quadcopter can be generated.
The first step is to set up the governing equation for the system \eqref{eq:starting_equation}, and rearrange it so that the right side is in terms of the second derivative of the state \eqref{eq:rearranged_starting_equation}.


\begin{equation} \label{eq:starting_equation}
M(q)\ddot{q} + C(q,\dot{q})\dot{q} + g(q) = B(q)u
\end{equation}

\begin{equation} \label{eq:rearranged_starting_equation}
\ddot{q} = M(q)^{-1}(-C(q,\dot{q})\dot{q} - g(q) + B(q)u)
\end{equation}

The relationship in \eqref{eq:internal_starting_equation} allows $\ddot{q}$ to be zero.

\begin{equation} \label{eq:internal_starting_equation}
B(q)u = C(q,\dot{q})\dot{q} + g(q)
\end{equation}

\begin{equation} \label{eq:C}
C(q,\dot{q}) = 
\begin{bmatrix}
A_{x}|\dot{x}| & 0 & 0 & 0 & 0 & 0 \\
0 & A_{y}|\dot{y}| & 0 & 0 & 0 & 0 \\
0 & 0 & A_{z}|\dot{z}| & 0 & 0 & 0 \\
0 & 0 & 0 & 0 & \dot{\psi}(I_{zz} - I_{yy}) & 0 \\
0 & 0 & 0 & 0 & 0 & \dot{\phi}(I_{xx} - I_{zz}) \\
0 & 0 & 0 & \dot{\theta}(I_{yy} - I_{xx}) & 0 & 0 \\
\end{bmatrix}
\end{equation}

\begin{equation} \label{eq:g}
g(q) =
\begin{bmatrix}
0 \\
0 \\
mg \\
0 \\
0 \\
0 \\
\end{bmatrix}
\end{equation}

$u$ is a vector containing the thrust force and the rotational torque in the body frame. $B(q)$ is the transformation matrix of the force and torque from the body frame to the inertial frame. $\omega$ is a vector containing the squared angular velocities of each rotor.

\begin{equation} \label{eq:u}
u =
\begin{bmatrix}
u_{1} \\
u_{2} \\
u_{3} \\
u_{4} \\
\end{bmatrix}
=
A\omega
=
\begin{bmatrix}
k & k & k & k \\
0 & -lk & 0 & lk \\
-lk & 0 & lk & 0 \\
-b & b & -b & b \\
\end{bmatrix}
\begin{bmatrix}
\omega_{1}^{2} \\
\omega_{2}^{2} \\
\omega_{3}^{2} \\
\omega_{4}^{2} \\
\end{bmatrix}
\end{equation}

\begin{equation} \label{eq:b}
B(q) =
\begin{bmatrix}
\cos{\phi}\sin{\theta}\cos{\psi} + \sin{\theta}\sin{\psi} & 0 & 0 & 0 \\
\cos{\phi}\sin{\theta}\sin{\psi} - \sin{\theta}\cos{\psi} & 0 & 0 & 0 \\
\cos{\phi}\cos{\theta} & 0 & 0 & 0 \\
0 & 1 & 0 & 0 \\
0 & 0 & 1 & 0 \\
0 & 0 & 0 & 1 \\
\end{bmatrix}
\end{equation}

Expanding \eqref{eq:internal_starting_equation} and rearranging the equation in terms of $\omega$ results in \eqref{eq:Y}.

%TODO find a good letter name for w->u
\begin{equation} \label{eq:Y} 
\omega = Y(q,\dot{q})\hat{\theta} = A^{-1}B(q)^{-1}[C(q,\dot{q})\dot{q} + g(q)]
\end{equation}

The form of $Y(q,\dot{q})$ can be determined in \eqref{eq:Y_expanded} by pulling out a vector of the unknown constant parameters. The true values of these parameters is shown in \eqref{eq:parameters}.

%TODO may need to rearrange this equation to reflect the rotor orientation
\begin{equation} \label{eq:Y_expanded} 
Y(q,\dot{q})\hat{\theta} =
\begin{bmatrix}
a|\dot{x}|\dot{x} & b|\dot{y}|\dot{y} & c|\dot{z}|\dot{z} & c & 0 & -\dot{\phi}\dot{\psi} & -\dot{\phi}\dot{\theta} \\
a|\dot{x}|\dot{x} & b|\dot{y}|\dot{y} & c|\dot{z}|\dot{z} & c & -\dot{\theta}\dot{\psi} & 0 & \dot{\phi}\dot{\theta} \\
a|\dot{x}|\dot{x} & b|\dot{y}|\dot{y} & c|\dot{z}|\dot{z} & c & 0 & \dot{\phi}\dot{\psi} & -\dot{\phi}\dot{\theta} \\
a|\dot{x}|\dot{x} & b|\dot{y}|\dot{y} & c|\dot{z}|\dot{z} & c & \dot{\theta}\dot{\psi} & 0 & \dot{\phi}\dot{\theta} \\
\end{bmatrix}
\begin{bmatrix}
\hat{\theta_{1}} \\
\hat{\theta_{2}} \\
\hat{\theta_{3}} \\
\hat{\theta_{4}} \\
\hat{\theta_{5}} \\
\hat{\theta_{6}} \\
\hat{\theta_{7}} \\
\end{bmatrix}
\end{equation}

\begin{subequations} \label{eq:abc_definitions}
\begin{equation}
d = (\cos{\phi}\sin{\theta}\cos{\psi} + \sin{\theta}\sin{\psi})^{2} + (\cos{\phi}\sin{\theta}\sin{\psi} - \sin{\theta}\cos{\psi})^{2} + (\cos{\phi}\cos{\theta})^{2}
\end{equation}
\begin{equation}
a = (\cos{\phi}\sin{\theta}\cos{\psi} + \sin{\theta}\sin{\psi}) / d
\end{equation}
\begin{equation}
b = (\cos{\phi}\sin{\theta}\sin{\psi} - \sin{\theta}\cos{\psi}) / d
\end{equation}
\begin{equation}
c = (\cos{\phi}\cos{\theta}) / d
\end{equation}
\end{subequations}

%TODO may need to change this slightly to reflect the rotor orientation
\begin{equation} \label{eq:parameters} 
\theta =
\begin{bmatrix}
A_{x}/4k \\
A_{y}/4k \\
A_{z}/4k \\
mg/4k \\
(I_{zz}-I_{yy})/2kl \\
(I_{xx}-I_{zz})/2kl \\
(I_{yy}-I_{xx})/2kl \\
\end{bmatrix}
\end{equation}

The adaptive controller and parameter update equations are shown in \eqref{eq:adaptive_control_equation} and \eqref{eq:adaptive_update_equation} respectively.
$K$ is the control gain matrix \eqref{eq:gain_matrix}, $T_{R}$ is the transformation from task space to rotor space \eqref{eq:rotor_transform}, $e$ is the state error, and $L$ is the learning rate.
The value of $L$ can either be a constant or a 4x4 matrix.
For simplicity a constant value of one is used.

%TODO need to mention and reference what K and e are in the text
\begin{equation} \label{eq:adaptive_control_equation}
\omega = Y(q,\dot{q})\hat{\theta} + T_{R}Ke
\end{equation}

\begin{equation} \label{eq:adaptive_update_equation}
\dot{\hat{\theta}} = LY^{T}T_{R}Ke
\end{equation}


\section{Controller Model}

%[show one model of the controller, the original one, talk about how it works]
%[something about how it is a target controller, rather than path]
%[maybe put some bit explaining the rationale behind the gains? Stuff like what increasing and decreasing specific gains does (might be better to just have this in the PID control section above)]

The goal of the controller is to actuate the quadcopter in such a way that it travels to a desired position in a reasonable amount of time. 
This position is specified as a set of x, y, and z coordinates and a particular yaw direction. 
The controller is given the state error of the quadcopter and its target, as well as the current linear and angular velocities of the quadcopter. 
It needs to use this information to generate a suitable control signal.

First, a PD controller generates a four-dimensional output signal in the space of possible quadcopter motions. 
The gain matrix that performs this operation can be seen in \eqref{eq:gain_matrix}. 
This signal multiplied by the matrix in \eqref{eq:rotor_transform}, is transformed into the four dimensional rotor velocity space to provide the actuation associated with the desired movement commands.
The design of this matrix depends upon the orientation of the rotor blades to the x and y axes. 
This transformation matrix assumes that the rotor axes are offset from the x-y axes by 45 degrees.

\begin{equation} \label{eq:gain_matrix}
K =
\setcounter{MaxMatrixCols}{12}
\begin{bmatrix}
0 & 0 & k_{2} & 0 & 0 & -k_{4} & 0 & 0 & 0 & 0 & 0 & 0 \\
0 & k_{1} & 0 & 0 & -k_{3} & 0 & -k_{5} & 0 & 0 & k_{7} & 0 & 0 \\
-k_{1} & 0 & 0 & k_{3} & 0 & 0 & 0 & -k_{5} & 0 & 0 & k_{7} & 0 \\
0 & 0 & 0 & 0 & 0 & 0 & 0 & 0 & -k_{6} & 0 & 0 & k_{8}
\end{bmatrix}
\end{equation}

\begin{equation} \label{eq:rotor_transform}
T_{R} = 
\begin{bmatrix}
1 & -1 & 1 & 1 \\
1 & -1 & -1 & -1 \\
1 & 1 & -1 & 1 \\
1 & 1 & 1 & -1
\end{bmatrix}
\end{equation}

\begin{equation} \label{eq:control_equation}
u = 
\begin{bmatrix}
w_{1}^{2} \\
w_{2}^{2} \\
w_{3}^{2} \\
w_{4}^{2}
\end{bmatrix}
= T_{R}K
\begin{bmatrix}
x \\
y \\
z \\
\dot{x} \\
\dot{y} \\
\dot{z} \\
\phi \\
\theta \\
\psi \\
\dot{\phi} \\
\dot{\theta} \\
\dot{\psi} \\
\end{bmatrix}
\end{equation}
%TODO make this equation look prettier on paper. There also might need to be tildes and things on the variables to show that they are state error.

Translation in the x and y directions is dependent on the states of both x and y as well as roll and pitch, because a roll in the quadcopter causes a component of thrust to be applied in the y direction, and a pitch causes a component of thrust to be applied in the x direction.
A delicate balance needs to be found between each of the gains in order to create a stable, functioning controller. 
The setpoint for each of the velocities as well as for roll and pitch is zero.

%[gain matrix figures]

%[talk about egocentric state as the error]
The state error is measured relative to the body frame because most sensors on a real quadcopter would return measurements relative to the sensor device itself, which is located on the quadcopter. Absolute measurements could still be obtained with a GPS device, but would typically be much less accurate and such devices are harder to use for fine tuned control. With the preferred manner, localized sensors, the state of the quadcopter can be defined relative to its target, making the state the same as the error.

%[talk about building the model in Nengo]
This controller model can be designed with a Nengo network, in which a 12-dimensional ensemble representing the state error can be projected to a 4-dimensional ensemble representing the desired control command. 
The transformation done through this projection will be by the 12x4 PD gain matrix of the controller. 
This 4-dimensional ensemble is then projected to another 4-dimensional ensemble which represents the four desired angular velocities of the quadcopter's rotors. 
This projection is done through a transformation by the 4x4 rotor matrix. 
This rotor velocity ensemble is connected to a node representing the physical quadcopter, which in turn feeds back into the state error ensemble. 
The network diagram is shown in \autoref{fig:NetBasic}. 
The network can be simplified further by multiplying the gain matrix with the rotor matrix to give a single transformation matrix from state error to rotor velocity. 
This simplified and functionally equivalent network is shown in Figure XXX.
The larger network is used in the remainder of this thesis because it is more explicit with how each signal is used and allows greater flexibility for design improvements and system debugging.

%[Figure of simple control model without adaptation]
%[Also figure of simplified model without task space representation]
\begin{figure}
\centering
\includegraphics[height=0.3\textheight]{./figures/NetBasic.png} %TODO get better quality figure
\caption{Basic Quadcopter Controller Network}
\label{fig:NetBasic}
\end{figure}

\section{Iterations}

%[show different iterations of the controller, explain strengths and weaknesses]
%[have diagrams of each, similar to the powerpoint presentation]
%[talk about Hyperopt parameter tuning]

The first iteration of the controller uses an adaptive population to influence the task space command. 
A projection from the adaptive population to the task population is modulated by the state error undergoing the control transform. 
The PES learning rule is applied to this connection, which seeks to build a transformation that minimizes the error coming in from this modulatory connection. 
The effect is similar to that of the I term in a PID controller, except that it uses all state information to come up with the integral gain, and can perform nonlinear transforms to accomplish this. Without this adaptive component, the quadcopter will have a large steady state error in the z direction, and will fall to the ground as soon as the simulation starts.

This model is very effective at learning the weight of the quadcopter and adapting to any external forces in the z direction. 
Gains were initially chosen manually by surveying other quadcopter models and trying a couple gains until reasonable results were found [reference v-rep quadcopter, and a few other places?]
%[[put in some plots showing adaptation speed at simulation start, talk about parameters that change this speed -> maybe show all gains used with the plot?]]

When an external force is applied in the horizontal plane, this controller does a very bad job of correcting from it. 
The quadcopter will settle to a stable point in space with a constant offset from the target location. 
This behaviour results from how the error is specified:
since both position and angle gains are being combined into the same dimension to produce the task space error, multiple pairs of measurements will produce the same error. 
That is, the adaptive population is being given an effective error reading of zero even though the quadcopter is not at the desired position. 
For example, if wind is producing a force in the x-direction, in order for the quadcopter to remain stationary, it must have a pitch angle that allows its thrust to compensate for both the gravitational force and the translational force.

%[[show FBD of quadcopter with pitch angle and forces balancing]]

Since the quadcopter has a non-zero pitch, the control signal produced by this pitch multiplied by its gain will be non-zero. 
In order to have zero error, the product of the x-position error and its gain must match this value. 
The quadcopter will consequently move away from the target x-position in order to maintain zero control error, an undesirable side-effect of this controller design. 
In fact, there is a whole space of angle-position measurement pairs that produce zero error.
A visualization of this space is depicted in \autoref{fig:ZeroErrorSurface}. 
The reason this controller works so well under normal operation despite this flaw is that only one point in that space is ever stable at any particular time. 
When the only force acting on the quadcopter is gravity, that point is at the target location with roll and pitch angles of zero radians. 
As external forces are applied along the horizontal plane, this stable point shifts to new locations.

%[[show diagram with the cone of 0 error states]]
\begin{figure}
\centering
\includegraphics[height=0.3\textheight]{./figures/ZeroErrorSurface.png} %TODO get better quality figure
\caption{Zero Control Signal Region}
\label{fig:ZeroErrorSurface}
\medskip
\small %TODO make the formatting of subtext nicer
This surface indicates the region of the quadcopter state where the control output is zero.
The target position is at [0,0] and the angle shown is a combination of the pitch and roll angles where $angle = \sqrt{\phi^{2} + \theta^{2}}$.
The quadcopter is only ever stable at one point on this surface.
The location of that point is dependent on the external forces acting on the quadcopter.
Under only the influence of gravity, the stable point is at the center of [0,0], but when an external force is applied along the horizontal plane, the stable point shifts in the direction of that force by an amount proportional to the magnitude of that force.
The slope of this surface is the ratio of the linear and angular P gains of the controller, $k_{1}/k_{5}$.
\end{figure}


One way to overcome this flaw is to modify the error signal that the adaptive population uses. 
The only important state variables that need to match the target are x,y,z position and yaw angle. 
The state velocities should all be zero as well. 
Since roll and pitch are not supposed to be controlled, they should really be left out of the error that the adaptive population uses since they don't correspond to an actual error in the desired state. 
This information is still important to allow the quadcopter to fly properly, and it needs to know roll and pitch information for stable flight. If this information is removed from the controller entirely it cannot fly.

One solution is to omit the roll and pitch information only from the adaptive population, while still using it for the baseline controller. 
This is done by generating a separate gain matrix to use on the modulatory connection to the learned transformation. 
The new network diagram of the model is shown below in \autoref{fig:NetAdaptiveTransform}. 
The form of the gain matrix for the modulatory connection is shown in \autoref{eq:adaptive_gain_matrix}.

\begin{equation} \label{eq:adaptive_gain_matrix}
K_{a} =
\setcounter{MaxMatrixCols}{12}
\begin{bmatrix}
0 & 0 & k_{a2} & 0 & 0 & -k_{a4} & 0 & 0 & 0 & 0 & 0 & 0 \\
0 & k_{a1} & 0 & 0 & -k_{a3} & 0 & 0 & 0 & 0 & 0 & 0 & 0 \\
-k_{a1} & 0 & 0 & k_{a3} & 0 & 0 & 0 & 0 & 0 & 0 & 0 & 0 \\
0 & 0 & 0 & 0 & 0 & 0 & 0 & 0 & -k_{6} & 0 & 0 & k_{8}
\end{bmatrix}
\end{equation}
%TODO maybe the roll/pitch velocities should still be kept here

%[[show network diagram with adaptive transform]]
\begin{figure}
\centering
\includegraphics[height=0.3\textheight]{./figures/NetAdaptiveTransform.png} %TODO get better quality figure
\caption{Quadcopter Controller Network with Adaptive Transformation}
\label{fig:NetAdaptiveTransform}
\end{figure}


\subsection{Gain Tuning}

This controller is now able to adapt to external horizontal forces, but normal flight is much less stable and prone to overshooting targets. 
One possible reason for this is that the gains might need to be re-tuned to work with this new controller setup. 
There are 12 different gains in this controller, which makes tuning them manually a difficult and laborious task.
This is a task well suited for automated parameter optimization. 
The tool that I chose to use for this is Hyperopt \cite{bergstra2013hyperopt}. 
Hyperopt is a python package designed to perform parameter optimization over a search space for a given function. 
As long as you can set your problem up in a way that it becomes a function that takes any number of parameters and returns a single error metric to be minimized, it can be used with Hyperopt. [[blather on about how hyperopt builds a surrogate model based on the problem, and how it has the option for smarter than random methods for picking points, as well as interesting ways to set up the search space]]

%TODO better title here??
\subsection{Using Hyperopt}

In order to use Hyperopt, the controller model must be encapsulated in some function that returns a useful metric of how well the controller works. This was done by creating a set of target points for the quadcopter to move through over a period of time and an additional population computing a scalar measure of the error of the quadcopter from the target. This error metric was taken to be a weighted Euclidean norm of each of the dimensions of the state. States that were deemed more important (such as x,y, and z position) were given higher weights in this calculation. States that were less important (such as roll and pitch) were given lower weights. 
Paired with this error was a status signal of whether or not the quadcopter should be at the target by now, or if it is okay that it is still in flight. 
As the quadcopter cannot be expected to move instantaneously between targets, it should not be penalized for having an error when it was just told to move to a different location. 
This status signal is set to 0 right after a new target is required, and then set to 1 again around the time when the quadcopter is expected to have reached the new target. The error at each time step is multiplied by this status signal before being recorded. 
The delay in switching back on the status signal can be used to give an indication of how important it is that the quadcopter reaches the target quickly. 
A longer delay means the optimization will find parameters that allow the quadcopter to reach the target very precisely with little oscillations, but could take a long time. 
A shorter delay will prefer controllers that get to the target quickly, but may overshoot, have a steady state error, or jitter once they reach the target.

Model creation, running the physics simulator, sending the target commands, stopping the physics simulator, and returning the error metric are all encapsulated in a single Python function. 
This function was then given to Hyperopt, which ran it many times and kept track of the parameters used for the best result. There were about 3000 [[check this!!]] evaluations of different parameter sets, and each run took about 26 seconds [[check this!!]]. 
Hyperopt is able to go through a set number of evaluation points in one run, and then pick up where it left off in a separate run. This allowed these runs to be completed overnight over the course of multiple nights to get the final results, shown in the Hyperopt column of \autoref{table:hyperopt_gains}. 
These are by no means the optimal gain parameters, as the function being optimized does not represent the full operational space of the controller, and the method for generating the error metric was very simplistic. 
The more complex the set of targets in the objective function, the longer each run will take to complete, meaning less of the parameter space can be explored in the same amount of time. 
A tradeoff had to be made between the amount of data to evaluate and the quality of each data point. 
Nevertheless, these gains turned out to produce a controller that works exceptionally well and is faster, more responsive, and more accurate than the previous controller.
The performance is further improved by augmenting the gains found through Hyperopt with hand-tuned gains.
These tweaks to the gains were done empirically to produce stronger performance in particular areas of flight. %TODO might need to expand on this / make it make more sense
These final gains are shown in the last column of \autoref{table:hyperopt_gains}.

%[[show a diagram or table of the target points used in the hyperopt objective function?]]

\begin{table}
\caption{Gain Values Obtained Through Hyperopt} \label{table:hyperopt_gains}
\begin{center}
\begin{tabular}{| l | l | l | l |}

\hline
\textbf{Gain} & \textbf{Original} & \textbf{Hyperopt} & \textbf{Hybrid} \\ \hline
$k_{1}$ & 0.22 & 0.4335 & 0.4335 \\ \hline
$k_{2}$ & 2.00 & 3.8617 & 8.0000 \\ \hline
$k_{3}$ & 0.47 & 0.5388 & 0.5388 \\ \hline
$k_{4}$ & 1.65 & 4.6702 & 6.6000 \\ \hline
$k_{5}$ & 3.80 & 2.5995 & 2.5995 \\ \hline 
$k_{6}$ & 10.0 & 0.8029 & 6.4230 \\ \hline 
$k_{7}$ & 1.95 & 0.5990 & 0.5990 \\ \hline 
$k_{8}$ & 3.16 & 2.8897 & 11.5589 \\ \hline 
$k_{a1}$ & 0.0063 & 0.0262 & 0.0262 \\ \hline 
$k_{a2}$ & 10.0 & 48.2387 & 26.00 \\ \hline 
$k_{a3}$ & 0.0150 & 0.0276 & 0.0276 \\ \hline 
$k_{a4}$ & 8.25 & 34.9626 & 21.45 \\ \hline 


\end{tabular}
\end{center}
\end{table}


%[talk about adaptive gains not needing to match regular ones, and go into how hyperparameter optimization is useful here. Talk about building an objective function to test parameter settings, and how it is not perfect and can be improved]


\subsection{Improving Adaptation to Horizontal Forces}

There is still much room for improvement on this controller. 
While it can adapt to horizontal forces, it does so quite slowly, as can be seen in \autoref{fig:PlotStandard}. 
This is because the adaptive component of the controller is always fighting against the command given by the standard component. 
There is an implicit target roll and pitch angle of zero radians that the controller is trying to achieve even if that is not the correct angle to be at. 
One approach to overcome this is to have a second adaptive population that tries to learn what roll and pitch angle will allow the quadcopter to be stationary, and then feed these angles to the basic controller. 
This way the controller will no longer be actively trying to bring the quadcopter away from its setpoint, yet still has the angle information required for it to be able to fly. 
A network diagram of this controller is shown in \autoref{fig:NetAngleAdapt} and the performance of this controller is shown in \autoref{fig:PlotAngleAdapt}.

\begin{figure}
\centering
\begin{subfigure}[t]{0.4\textwidth}
\includegraphics[height=0.3\textheight]{./figures/StandardWindX.png} %TODO get better quality figure
\caption{Displacement by Wind}
\end{subfigure}
\begin{subfigure}[t]{0.4\textwidth}
\includegraphics[height=0.3\textheight]{./figures/StandardWindY.png}
\caption{Trajectory to Target}
\end{subfigure}
\caption{Performance of Basic Adaptive Controller}
\label{fig:PlotStandard}
\end{figure}

\begin{figure}
\centering
\includegraphics[height=0.3\textheight]{./figures/NetAngleAdapt.png} %TODO get better quality figure
\caption{Quadcopter Controller Network with Angle Correction}
\label{fig:NetAngleAdapt}
\end{figure}

%[[show plots of angle adapt controller]]
\begin{figure}
\centering
\begin{subfigure}[t]{0.4\textwidth}
\includegraphics[height=0.3\textheight]{./figures/AngleCorrectionWindX.png} %TODO get better quality figure
\caption{Displacement by Wind}
\label{fig:PlotAngleAdaptA}
\end{subfigure}
\begin{subfigure}[t]{0.4\textwidth}
\includegraphics[height=0.3\textheight]{./figures/AngleCorrectionWindY.png}
\caption{Trajectory to Target}
\label{fig:PlotAngleAdaptB}
\end{subfigure}
\caption{Performance of Angle Correction Controller}
\label{fig:PlotAngleAdapt}
\end{figure}

%TODO figure out a better title for this section
\subsection{Shortcomings of the Error Signal}

There is still a problem with the implementation of this controller. 
Since the adaptive component is driven by state error, as soon as the target is moved to a new location, a large error will be produced. 
This error will cause the adaptive transformation to change, even if it was already at its optimal configuration. 
This can cause problems in control, such as the overshoot and oscillations seen in \autoref{fig:PlotAngleAdaptB}. 
The summation of the control signal of the underlying controller and the signal from the adaptation to the error creates a control signal that is too large for the desired performance. 
To account for this, a time delayed filter is applied to a population representing the target location. 
This filtered location is then subtracted from an unfiltered location and stored in a new population. The signal coming from this new population can be used to inhibit the adaptive populations. 
If the target hasn't moved for a while, the value being projected from this population [[give it some name to use!!]] will be close to zero, meaning there is no inhibition. 
If the target is suddenly moved, this population will output the difference between the two target positions and inhibit the adaptation in any dimension that has a difference. 
Over time the filtered target will start to match the unfiltered target, and the extent in which the angle adaptive population can adapt to any errors increases. 
The network diagram for this model can be seen in \autoref{fig:NetTMC} and the performance of this controller is shown in \autoref{fig:PlotTMC}.

%TODO should target following go anywhere?
%It's performance on the target following task can be seen in Figure XXX.

%[[put in network diagram, as well as one plot of performance]]
\begin{figure}
\centering
\includegraphics[height=0.3\textheight]{./figures/NetTMC.png} %TODO get better quality figure
\caption{Quadcopter Controller Network with Target Modulated Control}
\label{fig:NetTMC}
\end{figure}

\begin{figure}
\centering
\begin{subfigure}[t]{0.4\textwidth}
\includegraphics[height=0.3\textheight]{./figures/AlloBothTMCWindX.png} %TODO get better quality figure
\caption{Displacement by Wind}
\end{subfigure}
\begin{subfigure}[t]{0.4\textwidth}
\includegraphics[height=0.3\textheight]{./figures/AlloBothTMCWindY.png}
\caption{Trajectory to Target}
\end{subfigure}
\caption{Performance of Filtered Angle Correction Controller}
\label{fig:PlotTMC}
\end{figure}

This controller performs extremely well in both operation under normal conditions and in the presence of unknown external forces.

%TODO talk about the allocentric version too

% This figure will be the same as the non-allocentric version, just with less dimensions in the state, so probably don't need a new figure for it
%\begin{figure}
%\centering
%\includegraphics[height=0.3\textheight]{./figures/NetAlloTMC.png} %TODO get better quality figure
%\caption{Quadcopter Controller Network with Allocentric Target Modulated Control}
%\label{fig:NetAlloTMC}
%\end{figure}

%TODO this section may need a new name
\section{Implementation}

%[talk about how Nengo is connected to V-REP?]
%[Have a diagram about information flow?]
%[talk about the remote Api, setting flags in V-REP lua scripts, setting signals, sending plotting data. Talk about the run_model script and specific target patterns (or put that under experiments?)]

Each of these controllers is implemented as a Nengo model in Python. 
The main communication channel between V-REP and Nengo is the Quadcopter Node. 
This Node contains a callable Python class that manages the communication between the two systems at every time step and from the Nengo network's point of view represents the entirety of the physics simulation. 
Connections can be made to and from this node just like any other Node in a Nengo network. 
A connection to the V-REP remote API server is established when this Node is created. 
This node outputs the 6 dimensional state of the quadcopter, the 6 dimensional derivative of the state of the quadcopter, as well as the 6 dimensional state of the target. 
These values are obtained by making a remote API call to read these values from the current state of the simulation. The input to the Quadcopter Node is a 4 dimensional signal representing the velocity commands to give to each of the four rotors. 
These commands are packed into a string signal and sent to the V-REP simulation. 
A Lua script is run within V-REP each time step, and this script unpacks these velocity commands. 
It then issues them to the physical quadcopter model. The physics engine calculates the appropriate forces and torques that will be applied to the quadcopter as well as the resulting changes in state (position, velocity, orientation, and angular velocity) after one time step. 
This new state will now be read by the Nengo script in its next time step.

Nengo is run with a time step of 1ms, as this is the standard time step for most models and is sufficient for modelling spike timing effects in populations of LIF neurons while still running at a reasonable speed on most computer processors. 
The V-REP simulation on the other hand is run at a 10ms time step. 
Ideally it would also be run at 1ms, but the simulator has trouble rendering and making calculations that quickly. 
To account for this, Nengo only communicates with V-REP every 10 of its time steps. 
A synchronization trigger command is also issued every 10 time steps from Nengo. The V-REP simulation will only move forward one time step each time this trigger is issued, and Nengo will pause simulation until V-REP has completed this time step. 
This ensures that the timing of each simulation is always synchronized. 
There is some processing overhead in ensuring the synchronization of these two systems, but it is worth it for the improved accuracy of the overall simulation. 

%======================================================================
\chapter{Simulations and Results} \label{chap:analysis}
%======================================================================

\section{Experiments}

%[Run different models in different situations, PD, PID, Adaptive, (possibly different adaptive models)]
%[have some measure of goodness to run against -> measure of error throughout a run, -> speed of adaptation to picking up boxes or wind tunnel]
%[also put in an adaptive controller that is either allocentric or context sensitive, see if it does better on those measures]
%[lots of graphs and pretty pictures, maybe even a screenshot or two]

To get a sense of how well the neural adaptive controller performs, some reference implementations were created to use as benchmarks. 
Five different non-neural controllers were used, a standard PD controller, a standard PID controller, an improved PID controller, an improved PID controller with a faster integral gain, and an adaptive controller. [fix this sentence up, or just put them all in a table with a description].
% doublecheck that these are the ones you actually want to use, maybe narrow it down to only recording data for a few of them
A couple iterations of the neural adaptive controller are used in the benchmarking. 
These are the standard adaptive controller, the modified error adaptive controller, the angle corrective controller, the target modulated angle corrective controller, the allocentric controller, and the context sensitive controller. [just stick these in a table with descriptions as well] %TODO fix and re-work this sentence
The controllers used in the benchmarking are listed in \autoref{table:controller_models}

%[[show the math behind each of the controllers, mention the gravity compensation term used to account for the mass of the quadcopter, and how it is not needed in the adaptive models or PIDt. Have block diagrams or nengo diagrams of each]]

For each reference implementation, a gravity compensation term had to be calculated and applied to the controllers in order to obtain reasonable performance. 
The magnitude of this term was determined empirically and is proportional to the mass of the quadcopter. 
The rotor velocity signal is the sum of the gravity compensation term and the output of the controller.
The adaptive controllers do not need this extra term as they are able to learn how to compensate for the effects of gravity very quickly.

\begin{table}
\caption{Controller Models} \label{table:controller_models}
\begin{center}
\begin{tabular}{| l | p{9cm} |}

\hline
\textbf{Task} & \textbf{Description} \\ \hline
PD & Standard PD controller with gravity compensation term \\ \hline
PID & Standard PID controller with gravity compensation term \\ \hline
PIDt & PID controller where the error signal for the I term is in control space rather than state space \\ \hline
PIDtf & PIDt controller with a greater integral gain \\ \hline
Adaptive &  \\ \hline %TODO should this even be here? run some tests with it if you have time
Neural Adaptive & Simple adaptive neural controller. The network diagram can be seen in \autoref{fig:NetAdaptiveTransform} \\ \hline
Neural Adaptive Angle Correction & Adaptive neural controller with an additional adaptive neural population for determining stable roll and pitch. The network diagram can be see in \autoref{fig:NetAngleAdapt} \\ \hline
Neural Adaptive TMC & The difference between the current target state and a time delayed target modulates the error signal that drives the learning. The network diagram can be seem in \autoref{fig:NetTMC} \\ \hline
Neural Adaptive Allocentric TMC & The same as above, but absolute position and orientation information is projected into the adaptive populations along with the relative error to the target. The network diagram can be seem in \autoref{fig:NetTMC} \\ \hline

\end{tabular}
\end{center}
\end{table}

% TODO: this subsection should really come before the first experiments. May need to reword it so it makes sense to go there
\subsection{Metrics}

Three metrics are used to evaluate performance on these tasks. 
The first is the Root Mean Squared (RMS) error of the difference between the quadcopter's current state and its target state.
The quadcopter's state consists of position, velocity, orientation, and angular velocity.
These state variables are combined by computing the length of the resulting 12 dimensional error vector to produce a single quantity.
This value is calculated at each time step for the duration of the run and then averaged by the number of time steps in the run.
For point to point control this error is sometimes not the most informative because as soon as the target point has changed a large error value will be recorded even if the quadcopter is moving optimally towards its target.

The second metric is a modified version of the RMS error designed to take the desired trajectory into account.
This works by ignoring any error along the direction to the target as long as the current velocity is also in that direction.
It also ignores any error caused by the velocity in the correct direction as long as the quadcopter is not currently at its target.
There are also weights placed on specific types of errors to reflect an increased desire to minimize those errors.
For example, errors caused by overshooting the target are weighted more heavily.

The third metric is the time taken for the quadcopter to reach its target within a particular tolerance. 
This metric favours controllers that can reach the target quickly.
This metric does not worry about overshoot and non-optimal trajectories as long as steady state is achieved at the target in the end.
The particular tolerance chosen for these experiments is maintaining an RMS error of less than 0.001 for a duration of one second.



% TODO: maybe get a better title for this
\subsection{Benchmarks for Simple Environments}

A series of simple point to point control tasks are used to give an indication of performance. 
These tasks are: movement in the vertical direction, movement in the horizontal direction, rotation about the yaw axis, and horizontal movement into a wind tunnel. %, and a movement that requires a change in x, y, and z position as well as yaw angle. 
These tasks are summarized in \autoref{table:tasks} below.
Experimental results obtained from the reference controllers and the neural adaptive controllers are shown in \autoref{fig:benchmark_simple_reference} and \autoref{fig:benchmark_simple_neural} respectively. %TODO possibly put these into one figure, and make it easier to read - also need to make the scaling clear to compare neural to benchmark. Possibly have another figure with just the best neural vs the best reference

%[[Show table with different movement types, along with brief descriptions (i.e. move 2 units in the y direction into wind tunnel with 1 Newton of force in the x direction)]]
%[[maybe have multiple distances for each, and then average the RMS error, and report that for each model on the same graph? Should also have at least one plot from a single run, but its probably easiest to do massive comparisons with a single RMS number]]

%TODO fix up this table
\begin{table}
\caption{Benchmark Tasks} \label{table:tasks}
\begin{center}
\begin{tabular}{| l | p{9cm} |}

\hline
Task & Description \\ \hline
Vertical & Fly upwards (between one and five meters) \\ \hline
Horizontal & Fly in the x direction (between one and five meters) \\ \hline
Rotation & Rotate about the yaw axis (between 45 and 135 degrees) \\ \hline
Wind & Fly in the y direction (between one and five meters). Enter a wind tunnel after 0.75 meters. This wind tunnel exerts a force of 0.6 N in the x direction \\ \hline

\end{tabular}
\end{center}
\end{table}


\begin{figure}
\centering
\includegraphics[width=0.45\textwidth]{./figures/BenchmarkSimpleReferencePRMS.png} %TODO get better quality figure
\includegraphics[width=0.45\textwidth]{./figures/BenchmarkSimpleReferenceTime.png} %TODO get better quality figure
\caption{Performance on Benchmarks from Reference Controllers}
\label{fig:benchmark_simple_reference}
\end{figure}

\begin{figure}
\centering
\includegraphics[width=0.45\textwidth]{./figures/BenchmarkSimpleNeuralPRMS.png} %TODO get better quality figure
\includegraphics[width=0.45\textwidth]{./figures/BenchmarkSimpleNeuralTime.png} %TODO get better quality figure
\caption{Performance on Benchmarks from Neural Adaptive Controllers}
\label{fig:benchmark_simple_neural}
\end{figure}


%[put some stuff in here with noise as well?]

%[also do experiments moving throughout a weird force field, and seeing improvement the more times it does it. This will have to be the allocentric model]



% TODO: figure out a good name for this section
\subsection{Benchmarks for Complex Environments}

The performance of the adaptive controller is quite strong, but it's performance is still relatively close to that of the modified PID controllers. 
Where the adaptive controller really excels is when there are external forces being applied that are functions of the system state. 
The adaptive controller is designed to be able to account for both linear and nonlinear functions of the system state.

To quantify the performance of the quadcopter controllers under the influence of these more interesting forces, a new set of benchmark tasks is used.
These tasks are listed in \autoref{table:forcingfunctions} below. %maybe just reference the forcing function table here? Have both places reference the same table

%put table with strange forces here, include downward force proportional to horizontal velocity, sideways force proportional to position/velocity, etc.


Performance of the Neural Adaptive controller is compared to the reference controller on these tasks.
The results are displayed in \autoref{fig:benchmark_complex}. 

%TODO organize the layout better and get better quality figures
\begin{figure}
\centering
\includegraphics[width=0.45\textwidth]{./figures/BenchmarkComplexFF0PRMS.png}
\includegraphics[width=0.45\textwidth]{./figures/BenchmarkComplexFF0Time.png}
\includegraphics[width=0.45\textwidth]{./figures/BenchmarkComplexFF1PRMS.png}
\includegraphics[width=0.45\textwidth]{./figures/BenchmarkComplexFF1Time.png}
\includegraphics[width=0.45\textwidth]{./figures/BenchmarkComplexFF3PRMS.png}
\includegraphics[width=0.45\textwidth]{./figures/BenchmarkComplexFF3Time.png}
\includegraphics[width=0.45\textwidth]{./figures/BenchmarkComplexFF5PRMS.png}
\includegraphics[width=0.45\textwidth]{./figures/BenchmarkComplexFF5Time.png}
\includegraphics[width=0.45\textwidth]{./figures/BenchmarkComplexFF6PRMS.png}
\includegraphics[width=0.45\textwidth]{./figures/BenchmarkComplexFF6Time.png}
\caption{Performance on Benchmarks under different External Force Conditions}
\label{fig:benchmark_complex}
\end{figure}

%Fill the rest of this subsection with the results of running models in these environments
% could just put things in tables instead of having plots everywhere, or make the plots smaller so they don't take up an obnoxious amount of space

% this is where the loop figures will be. Talk about the controller getting better over time by learning the space
\subsection{Improvement over Time} %TODO better title for this section

The previous experiments only track performance for a single short run. 
The Neural Adaptive Controller is able to learn how to move throughout its environment better over time.
An example of this ability is shown in \autoref{fig:loop_path}.
The quadcopter is commanded to move back and forth between two points on the x-y plane ([0,0] and [3,0]) with a 5 second delay between each new command.
An external force is being applied to the quadcopter along the y-direction proportional to its x-direction velocity.
This causes the path of the quadcopter to become curved rather than a straight line.
Over time, the Neural Adaptive Controller begins to compensate for this external force, and the path between the two points starts to converge towards a straight line.
The non-adaptive controllers show no such improvement over time.

%TODO include numeric RMS values to accompany the plots
\begin{figure}
\centering
%\includegraphics[height=0.15\textheight]{./figures/allo_both_tmc_loop.png}
%\includegraphics[height=0.15\textheight]{./figures/pd_loop.png}
%\includegraphics[height=0.15\textheight]{./figures/pidtf_loop.png}
\includegraphics[width=0.32\textwidth]{./figures/allo_both_tmc_loop.png}
\includegraphics[width=0.32\textwidth]{./figures/pd_loop.png}
\includegraphics[width=0.32\textwidth]{./figures/pidtf_loop.png}
\caption{Path of Quadcopter between Two Points with External Forces} %mention something about how the experiment was done here? it was 100 seconds total with 5 seconds before the point switches
\label{fig:loop_path}
\end{figure}


% This subsection will talk about how pre-training for different amounts of time on particular force functions effects the performance 
\subsection{Effects of Training Time}

Adaptive controllers learn over time.
Longer operation contributes to a better internal representation of the environment which in turn increases the performance that the controller can achieve.
As such, it useful to characterize by how much these controllers improve their performance as a function of time.

Training is done by running the controller in a particular environment for a period of time and recording the values of the learned decoders. % maybe explain this more??
These decoder values can then be loaded into the controller at the start of a new simulation, and the new controller will effectively start off with all of the information that the old controller had learned.
By taking snapshots of the decoder values at particular intervals throughout the training run, the new simulations can be run with different levels of pre-training to compare performance.
Each training run is completed with a different forcing function being applied to the quadcopter.
The particular forcing functions used in these experiments are detailed in \autoref{table:forcingfunctions}.
In order to explore the state space, the quadcopter is sent to a series of random target locations throughout the training run.
In the experiments shown below, 15 random targets were chosen during training.
These targets are sampled from a 6 by 6 area in the horizontal plane.
Every three seconds the quadcopter is commanded to move to the next target in the sequence. Once the end of the sequence has been reached, the sequence will start over again until the end of the training session. %TODO this is out of date, future runs have everything random, with no repeating of the sequence
Training sessions lasted 1000 seconds, with recordings taken every 10 seconds. %the recordings should really be on a more exponential scale, i.e. [0,5,10,50,100,500,1000]

\begin{table}
\caption{Forcing Functions} \label{table:forcingfunctions}
\begin{center}
\begin{tabular}{| l | p{9cm} |}
\hline
\textbf{Name} & \textbf{Description} \\ \hline
None & No external forces applied \\ \hline
Vertical Velocity & A downward force is applied proportional to the quadcopter's horizontal velocity \\ \hline
Horizontal Position & A force is applied in the y-direction proportional to the quadcopter's x-position \\ \hline
Horizontal Velocity & A force is applied in the y-direction proportional to the quadcopter's x-velocity \\ \hline
\end{tabular}
\end{center}
\end{table}
% could also add and try out forces dependent on the vertical direction, or even rotation

These trained controllers ran on the benchmark tasks from Section XXX.
The performance for different lengths of training time are shown in \autoref{fig:training_time} below.

%put the figure with the different forcing functions and training time here
\begin{figure} \label{fig:training_time}

\end{figure}

These results show a clear improvement in the controller performance after training.
The amount of training needed to get good results is relatively small. 
After about XXX seconds, the controller performance does not improve significantly.

%maybe show another figure with training done on more than 15 points? Highlight how exploring a large portion of the space is important


% does this even need to be a section? Results go hand in hand with experiments
\section{Results}

%[Talk about the results, give advantages and disadvantages of each]

While the neural adaptive controller did not perform strictly the best in all tasks, it had the strongest overall performance.
 
%======================================================================
\chapter{Discussion and Future Work} \label{chap:discussion}
%======================================================================
% maybe split the following sections into contributions and extensions?

% talk about what this thesis provides to the community. What is new about it
\section{Contributions}

%mention adaptability to unknown environments with the potential to be run on low power neuromorphic hardware (maybe something about how Moore's law is ending, and neuromorphic hardware could be the future)
%mention integrating Nengo with V-REP to allow more interesting biological models

The main contribution of this thesis is an adaptive control system for a quadcopter using simulated biological neurons.

A second contribution is the integration of the Nengo neural simulation software with the robotics and physics simulation capacity of V-REP.
This union allows complex neural models to be embodied in a physical environment in a straightforward manner. 

%TODO talk about noise here? Throw in one simulation with noise to say that it works
%TODO can also throw in one simulation of target following and mention how it works too
\section{Discussion}

\section{Future Work}

Further research could usefully explore improvements to this model in terms of both physical accuracy and additional capabilities.
The closer the model behaves to a real physical system, the more useful it will be in guiding the design of real world applications.
The more capabilities the model displays, the more variety in the applications it can be applied to.
%TODO fix this last sentence up a bit

\subsection{Adaptive Prediction System}

In addition to control, the inverse problem of predicting future states is important.
Integrating adaptive control with an adaptive method for system identification could lead to major improvements in overall system performance.

\subsection{Integrated Navigation and Planning System}

A useful extension of the adaptive controller would be to embed it within a larger navigation and planning system.
Currently the quadcopter will only fly in a direct path towards a specified point, but it would be useful for it to be able to detect obstacles in its path and calculate a new route to avoid those obstacles.
The adaptive population of neurons in the controller could be set up to be able to recognize certain obstacles and learn to avoid them in an effective manner.
An internal representation of the environment along with a memory for changes in the environment could be used to allow dynamic planning of optimal trajectories to a target.

% talk about modelling more complex things like blade flapping and vortex ring state
%TODO maybe these weird states could be learned? That would be cool.
\subsection{Detailed Modelling}



% talk about using camera for odometry, or ultrasonic/infrared for distance, etc
\subsection{Realistic Sensors}

The current implementation of the simulation is able to directly read the exact state information of the quadcopter.
A more realistic scenario would be to use a set of sensors that provide measurements that can be used to estimate the state.
Each of these sensors can have different noise properties and accuracy that can change depending on the situation.
The robustness of the adaptive control algorithm to changes in the sensor properties and estimation algorithm can be important to characterize when looking into purchasing components to implement the controller on physical hardware.


\subsection{Running on Physical Hardware}

The most practical extension to this work would be to move beyond simulation and run the control system on physical hardware.
%Many challenges will arise in this undertaking, including selection of the particular hardware platform, sensors to use, timing issues in the control algorithm.

% The \appendix statement indicates the beginning of the appendices.
\appendix

% Add a title page before the appendices and a line in the Table of Contents
\chapter*{APPENDICES}
\addcontentsline{toc}{chapter}{APPENDICES}
%======================================================================
\chapter[Link to the Source Code]{Source Code for Thesis Work}
\label{AppendixA}
% Tip 4: Example of how to get a shorter chapter title for the Table of Contents 
%======================================================================

All of the source code used to create the quadcopter controllers presented in this thesis is available for download at: https://github.com/bjkomer/masters-thesis

Instructions for downloading and installing the necessary software to run the simulations and benchmarks are also provided at this link.

%----------------------------------------------------------------------
% END MATERIAL
%----------------------------------------------------------------------

% B I B L I O G R A P H Y
% -----------------------

% The following statement selects the style to use for references.  It controls the sort order of the entries in the bibliography and also the formatting for the in-text labels.
\bibliographystyle{plain}
% This specifies the location of the file containing the bibliographic information.  
% It assumes you're using BibTeX (if not, why not?).
\cleardoublepage % This is needed if the book class is used, to place the anchor in the correct page,
                 % because the bibliography will start on its own page.
                 % Use \clearpage instead if the document class uses the "oneside" argument
\phantomsection  % With hyperref package, enables hyperlinking from the table of contents to bibliography             
% The following statement causes the title "References" to be used for the bibliography section:
\renewcommand*{\bibname}{References}

% Add the References to the Table of Contents
\addcontentsline{toc}{chapter}{\textbf{References}}

\bibliography{thesis}
% Tip 5: You can create multiple .bib files to organize your references. 
% Just list them all in the \bibliogaphy command, separated by commas (no spaces).

% The following statement causes the specified references to be added to the bibliography% even if they were not 
% cited in the text. The asterisk is a wildcard that causes all entries in the bibliographic database to be included (optional).
\nocite{*}

\end{document}
