% T I T L E   P A G E
% -------------------
% Last updated May 24, 2011, by Stephen Carr, IST-Client Services
% The title page is counted as page `i' but we need to suppress the
% page number.  We also don't want any headers or footers.
\pagestyle{empty}
\pagenumbering{roman}

% The contents of the title page are specified in the "titlepage"
% environment.
\begin{titlepage}
        \begin{center}
        \vspace*{1.0cm}

        \Huge
        {\bf Biologically Inspired Adaptive Control of Quadcopter Flight }

        \vspace*{1.0cm}

        \normalsize
        by \\

        \vspace*{1.0cm}

        \Large
        Brent Komer \\

        \vspace*{3.0cm}

        \normalsize
        A thesis \\
        presented to the University of Waterloo \\ 
        in fulfillment of the \\
        thesis requirement for the degree of \\
        Master of Mathematics \\
        in \\
        Computer Science \\

        \vspace*{2.0cm}

        Waterloo, Ontario, Canada, 2015 \\

        \vspace*{1.0cm}

        \copyright\ Brent Komer 2015 \\
        \end{center}
\end{titlepage}

% The rest of the front pages should contain no headers and be numbered using Roman numerals starting with `ii'
\pagestyle{plain}
\setcounter{page}{2}

\cleardoublepage % Ends the current page and causes all figures and tables that have so far appeared in the input to be printed.
% In a two-sided printing style, it also makes the next page a right-hand (odd-numbered) page, producing a blank page if necessary.
 


% D E C L A R A T I O N   P A G E
% -------------------------------
  % The following is the sample Delaration Page as provided by the GSO
  % December 13th, 2006.  It is designed for an electronic thesis.
  \noindent
I hereby declare that I am the sole author of this thesis. This is a true copy of the thesis, including any required final revisions, as accepted by my examiners.

  \bigskip
  
  \noindent
I understand that my thesis may be made electronically available to the public.

\cleardoublepage
%\newpage

% A B S T R A C T
% ---------------

\begin{center}\textbf{Abstract}\end{center}

This thesis explores the application of a biologically plausible learning method to quadcopter flight control.
%TODO finish the abstract

\cleardoublepage
%\newpage

% A C K N O W L E D G E M E N T S
% -------------------------------

\begin{center}\textbf{Acknowledgements}\end{center}

I would like to thank all the little people who made this possible.
\cleardoublepage
%\newpage

% D E D I C A T I O N
% -------------------

\begin{center}\textbf{Dedication}\end{center}

This is dedicated to the one I love.
\cleardoublepage
%\newpage

% T A B L E   O F   C O N T E N T S
% ---------------------------------
\renewcommand\contentsname{Table of Contents}
\tableofcontents
\cleardoublepage
\phantomsection
%\newpage

% L I S T   O F   T A B L E S
% ---------------------------
\addcontentsline{toc}{chapter}{List of Tables}
\listoftables
\cleardoublepage
\phantomsection		% allows hyperref to link to the correct page
%\newpage

% L I S T   O F   F I G U R E S
% -----------------------------
\addcontentsline{toc}{chapter}{List of Figures}
\listoffigures
\cleardoublepage
\phantomsection		% allows hyperref to link to the correct page
%\newpage

%TODO possibly use a glossary for the Nengo terms
% L I S T   O F   S Y M B O L S
% -----------------------------
% To include a Nomenclature section
 %\addcontentsline{toc}{chapter}{\textbf{Nomenclature}}
 %\renewcommand{\nomname}{Nomenclature}
 %\printglossary
 \addcontentsline{toc}{chapter}{\textbf{Glossary}}
 \renewcommand{\nomname}{Glossary}
 \makenomenclature
 \nomenclature{\textbf{Ensemble}}{A group of neurons representing a single vector}
 \nomenclature{\textbf{Node}}{}
 \nomenclature{\textbf{Network}}{A group of Ensembles, Nodes, Connections, and other Networks within a Nengo model}
 \nomenclature{\textbf{Connection}}{A link between the output of one Ensemble or Node to the input of another Ensemble or Node}
 \nomenclature{\textbf{Nengo}}{A Python software package that implements algorithms from the Neural Engineering Framework}
 \nomenclature{\textbf{Neural Engineering Framework}}{A set of methods for performing computations with simulated ensembles of neurons}
 %\nomenclature{Population}{} %TODO just use Ensemble everywhere to be clear
 \nomenclature{\textbf{Encoders}}{Functions applied to a vector to produce neural activities}
 \nomenclature{\textbf{Decoders}}{Weightings applied to neural activities to produce a vector}
 \nomenclature{\textbf{Synapse}}{}
 \nomenclature{\textbf{Tuning Curve}}{Response characteristics of a neuron}
 \printnomenclature
 \cleardoublepage
 \phantomsection % allows hyperref to link to the correct page
 \newpage

% G L O S S A R Y
% ---------------
%\addcontentsline{toc}{chapter}{\textbf{Glossary}}
%\printglossary
%\cleardoublepage
%\phantomsection

% Change page numbering back to Arabic numerals
\pagenumbering{arabic}

